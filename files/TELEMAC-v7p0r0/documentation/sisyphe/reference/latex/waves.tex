\section{Wave effects}\label{ch:waves}
\subsection{Introduction}
In coastal zones, the effect of waves superimposed to a mean current (wave-induced or tidal), modifies the bed structure. Due to the reduced thickness of the bed boundary layer, the bottom
shear stress is largely increases and the resulting sand transport rate is, in many cases, of
an order of magnitude greater than in the case of currents alone. Furthermore, the wave-generated ripples may also have a strong effect on the bed roughness and sand transport mechanisms. In \sisyphe, these effects can be incorporated into the numerical simulation when the keyword \texttt{EFFECT OF WAVES} is activated.

To compute sediment transport rates due to the action of waves, the wave height, period and direction need to be specified. This information can be provided from a fortran file, by the subroutine \texttt{condim\_sisyphe.f}, from a file containing the variables computed previously by the wave module, e.g. \tomawac~\cite{reftomawac}, or by internal coupling.

The wave orbital velocity $U_0$ is calculated by \sisyphe assuming linear theory:
\begin{equation*}
U_0=\frac{H_s \omega }{2 \sinh (kh)}, 
\end{equation*}
where $H_s$ is the significant wave height, $\omega = 2\pi/T_p$ is the intrinsic angular frequency, $k = 2\pi/L$ is the wave number, with $L$ the wave length. The wave number is calculated from the dispersion relation (neglecting surface tension and amplitude effects):
\begin{equation*}
\omega^2 = gk\tanh (kh). 
\end{equation*}

\subsubsection{Wave information}
The wave information, required as input parameters for the morphodynamic
computation, are the significant wave height $H_s$, the peak wave
period $T_p$ and the wave direction $\theta_w$. If $\theta_w$ is not provided, 
\sisyphe assumes the default value $\theta_w = 90^\circ$, which means that the
direction of propagation of the wave is parallel to the $x$-axis. When using the 
Bailard's transport formula, this variable must be always provided.

If the wave field is provided from a file containing the variables computed previously on the same mesh by the wave module, e.g. \tomawac, the keyword \texttt{WAVE FILE} must be specified in the \sisyphe steering file. This file must contain the wave variables \texttt{HM0}, significant wave height $H_s$ (\texttt{HM0}), wave peak period $T_p$ (\texttt{TPR5}) and mean wave direction $\theta_w$, relative to $y$-axis (\texttt{DMOY}).

\begin{bclogo}[couleur = blue!10, arrondi = 0.10, logo = \bcattention]{\textsf{Attention}}
For the keyword \texttt{WAVE FILE}, variables are given by the last record of the file.
\end{bclogo}

If the currents and wave field is provided from a file containing the variables computed previously on the same mesh by the currents and wave modules, e.g. \teldd and \tomawac, respectively, the keyword \texttt{HYDRODYNAMIC FILE} must be specified in the \sisyphe steering file. This file must contain the hydrodynamic variables (water height and velocity file) as well as wave variables \texttt{HM0}, significant wave height $H_s$ (\texttt{HM0}), wave peak period $T_p$ (\texttt{TPR5}) and mean wave direction $\theta_w$, relative to $y$-axis (\texttt{DMOY}). The users has also the choice of provide the wave variables from the subroutine \texttt{condim\_sisyphe.f}.  
 
\begin{bclogo}[couleur = blue!10, arrondi = 0.10, logo = \bcattention]{\textsf{Attention}}
For the keyword \texttt{HYDRODYNAMIC FILE}, variables are given by the last record of the file if the case is 
steady or by the time steps describing the tide or flood for the unsteady case. 
\end{bclogo}

%The (last time record of the) `wave file' issued from previous TOMAWAC
%computation is read by \sisyphe. In addition, the mean current issued from
%previous \tel computation is read from the `hydrodynamic file'.
%For unsteady flow conditions (time-varying velocity and wave field), the
%user should run Tomawac in an unsteady mode, and store all the wave and
%hydrodynamic relevant variables on the same input `hydrodynamic file'.

\subsection{Wave-induced bottom friction}

\subsubsection{Wave-induced bottom friction}
The maximum stress due to waves is calculated at each time step as a
function of the wave-orbital velocity $U_w$ by use of a quadratic
friction coefficient $f_w$ due to waves:
\begin{equation*}
\tau_w = \frac{1}{2}\rho f_w U_w^2. 
\end{equation*}
The wave friction factor $f_w$ is calculated as a function of relative
density:
\begin{equation*}
f_w = f_w \left( A_0/k_s \right), 
\end{equation*}
where $A_0= U_w/\omega$ is the orbital amplitude on the bed and $k_s$ the bed roughness.
In \sisyphe, the formule proposed by Swart \emph{et al.}~\cite{Swart} is provided
\begin{equation*}
f_w = \left\{\begin{array}{ll}
\exp \left(-6.0 + 5.2\left( \frac{A_0}{k_s} \right)^{-0.19}\right), & \quad \text{if } \frac{A_0}{k_s} > 1.57\\
0.30, & \quad \text{otherwise}
\end{array}
\right.
\end{equation*}

\subsubsection{Wave-current interaction}
For combined waves and currents, the wave-induced (mean and maximum) bottom
stresses are, in many cases, of an order of magnitude larger than in the case of currents
alone. Different models can be found in the literature to calculate the wave
and current bottom stresses $\tau_{cw}$, as a function of the bottom
shear stress due to currents only $\tau_c$ and the maximum shear
stress due to waves only $\tau_w$. Following Bijker~\cite{Bijker}:
\begin{equation}\label{eq:tauBijker}
\tau_{cw} = \tau_c + \frac{1}{2} \tau_w. 
\end{equation}

The mean $\tau_{mean}$ and maximum $\tau_{\max}$ shear stresses can
also be calculated following the Soulsby's method~\cite{Soulsby}. Non-dimensional
variables are defined:
\begin{equation*}
X=\frac{\tau_0}{\tau_0+\tau_w};\quad Y=\frac{\tau_{mean}}{\tau_0+\tau_w};\quad  
Z=\frac{\tau_{\max}}{\tau_0+\tau_w} 
\end{equation*}
% CHECK: mean == moy?
They can be parameterized as:
\begin{align*}
Y &= X\left(1+bX^p(1-X)^q\right),\\
Z &= 1+aX^m(1-X)^n 
\end{align*}
The various model coefficients ($a, b, m, n, p, q$) are empirical functions of
friction coefficients $f_w$ and $C_D$, and the wave-current angle $\phi$:
\begin{align*}
a &= (a_1 +a_2 \left|\cos \phi \right|^I)+(a_3+a_4 \left|\cos
\phi \right|^I)\log_{10}\left(\frac{2f_w}{C_D} \right),\\
b &= (b_1 +b_2 \left|\cos \phi \right|^J)+(b_3+b_4 \left|\cos
\phi \right|^J)\log_{10} \left(\frac{2f_w}{C_D } \right),\\
..., &   
\end{align*}
with similar expressions for $m, n, p$, and $q$. The fitted constants ($a_i$, $b_i$, $m_i$, $n_i$, $p_i$, $q_i$, $I$ and $J$) depend on the turbulence model selected. Table~\ref{tab:constants} shows the coefficients used in \sisyphe corresponding to the model of Huynh-Thanh and Temperville~\cite{Huynh}, for $I = 0.82$, $J = 2.7$.

\begin{table}[H]
 \centering
\begin{tabular}{ccccccc}
\hline
$i$  & $a_i$ & $b_i$ & $m_i$ & $n_i$ & $p_i$ & $q_i$ \\
\hline
$1$ & $-0.07$ &  $0.27$ & $0.72$ & $0.78$ & $-0.75$ & $0.89$ \\
$2$ & $1.87$ & $0.51$ & $-0.33$ & $-0.23$ & $0.13$ & $0.40$ \\
$3$ & $-0.34$ & $-0.10$ & $0.08$ & $0.12$ & $0.12$ & $0.50$ \\
$4$ & $-0.12$ & $-0.24$ & $0.34$ & $-0.12$ & $0.02$ & $-0.28$ \\
\hline
 \end{tabular}
\caption{Constants issued from the turbulence model of Huynh-Thanh and Temperville~\cite{Huynh}, see also Soulsby~\cite{Soulsby}, page 91.}
  \label{tab:constants}
\end{table}

\subsubsection{Friction coefficient under combined waves and current}
The characteristic shear stress to represent wave current interactions $\left\langle \left| \overrightarrow{\tau_{cw}} (t)\right| \right\rangle$, is related to the time-averaged mean velocity:
\begin{equation*}
\left\langle \left| \overrightarrow{\tau_{cw}} (t)\right| \right\rangle
=\rho f_{cw} \left\langle \left| \overrightarrow{U(t)} \right|^2
\right\rangle, 
\end{equation*}
where
\begin{equation*}
\left\langle \left| \overrightarrow{U(t)} \right|^2 \right\rangle =U_c^2 +\frac{1}{2} U_w^2 
\end{equation*}
According to Camenen~\cite{Camenen}, the characteristic shear stress is taken as a weighted average between $\tau_{mean}$ and $\tau_{\max}$:
\begin{equation*}
\left\langle \left| \overrightarrow{\tau_{cw}} (t)\right| \right\rangle
 = X\tau_{mean} + (1-X)\tau_{\max}, 
\end{equation*}
which is equivalent to:
\begin{equation*}
\left\langle \left| \overrightarrow{\tau_{cw}} (t)\right| \right\rangle
 = Y\tau_c + Z\tau_w. 
\end{equation*}
The final expression for the wave-current interaction factor is:
\begin{equation}\label{eq:fcw}
f_{cw} = \frac{Y\tau_c + Z\tau_w}{{U_c^2} + \frac{1}{2}{U_w^2}}. 
\end{equation}
This expression of the wave and current friction factor is used by Bailard~\cite{13} and Dibajnia and Watanabe~\cite{Dibajnia} to compute sand transport rates.

\subsection{Wave-induced ripples}
Equilibrium ripples are generally observed outside the surf zone. Their
dimensions can be predicted as a function of waves (orbital velocity $U_0$
and wave period $T=2\pi/\omega$), for a given uniform sediment diameter $d_{50}$. 
In \sisyphe, the procedure of Wiberg and Harris has been implemented~\cite{Wiberg}. This formulation is only applicable to oscillatory flow conditions and does not account for the effect of a superimposed mean current.

Ripples can be classified into three types depending on the value of the
ratio of wave orbital diameter to mean grain diameter, $D_0/d_{50}$,
with $D_0 = 2U_0/\omega$. The ripples wave length $\lambda$ and height $\eta$, can be estimated as follows: 

\begin{itemize}
\item Under moderate wave conditions (orbital regime), the ripple dimensions are proportional to $D_0$:
\begin{equation*}
\lambda = 0.62D_0, \quad \eta = 0.17\lambda.
\end{equation*}
\item For larger waves (anorbital regime), ripple dimensions scale with the sand grain diameter:
\begin{align*}
\lambda &= 535\,d_{50}, \\
\eta &= \lambda \exp \left(-0.095\left(\log\frac{D_0}{\eta} \right)^2
 + 0.042\log\frac{D_0}{\eta}-2.28\right). 
\end{align*}
\end{itemize}
The effect of ripples is to increase the bed roughness. Based on the Bijker's
formula, in \sisyphe it is assumed that:
\begin{equation*}
k_s = \max \left(\eta ,3d_{50} \right). 
\end{equation*}

The effect of the mean current superimposed to the waves is accounted by
introducing a correction factor $\alpha$ to the orbital velocity,
following Tanaka and Dang~\cite{Tanaka}:

\begin{equation*}
\alpha = 1 + 0.81\left(\tanh (0.3S_*^{2/3})\right)^{2.5}
\left(\frac{U}{U_w}\right)^{1.9}, 
\end{equation*}
with $S_* = d_{50}\sqrt{(s-1)g d_{50}}/4\nu$. 

\subsection{Wave-induced sand transport}
The effect of waves modifies the sand transport rates when it is superimposed to currents. 
This effect can be accounted by using one of the following wave sand transport formula programmed
in \sisyphe (keyword \texttt{BED-LOAD TRANSPORT FORMULA}), namely the Bijker's formula~\cite{Bijker}, the Soulsby-van Rijn's formula~\cite{Soulsby}, the Bailard's formula~\cite{13} and the Dibajnia and Watanabe's formula~\cite{Dibajnia}.

\begin{itemize}
\item \texttt{ICF = 4}: the Bijker's formula (1968) can be used for determining the \textbf{total transport
rate} $Q_s$ (bed-load + suspension), with its two components the bedload 
$Q_{sc}$ and the suspended load $Q_{ss}$ determined separately. %The
%development is made on the scalar variables, since Bijker's formula is an
%extension of a steady flow formula to account for the effect of the wave
%enhanced shear stress.

For bedload transport, Bijker extended the steady bed-load formula proposed by Frijlink (1952),
by adding the effect of the wave. In non-dimensional variables, the
bedload transport rate is:

\begin{equation*}
\Phi_b = b\theta_c^{0.5}\exp\left(-0.27\frac{1}{\mu \theta_{cw}}\right),
\end{equation*}

where $\theta_c$ is the non-dimensional shear stress due to currents
alone, $\theta_{cw}$ the non-dimensional shear stress due to wave-current
interaction, and $\mu$ is a correction factor which accounts for the effect
of ripples. The shear stress under combined wave and current is calculated
by Equation (\ref{eq:tauBijker}).

In the original formulation, the coefficient $b$ is $b=5$. Recent studies~\cite{17} showed
that $b=2$ provides better results when comparing with field and experimental data. Be default, in \sisyphe $b=2$ but this value can be modified with the keyword \texttt{B VALUE FOR THE BIJKER FORMULA}. The ripple factor correction $\mu$ is calculated in the same way as in the
Meyer-Peter-Mueller formula for currents only.

The suspended load transport is solved in a simplified manner, the
concentration profile is assumed to be in equilibrium. The
inertia effects related to the water column loading and unloading will then
in no way be modelled. Furthermore, no exchange takes place with the bed
load layer, therefore only the continuity of concentration is ensured. After depth-integration and by assuming a Rouse profile for the concentration
and a logarithmic velocity profile for the mean velocity profile, the
suspended load can be written as:
\begin{equation*}
Q_{ss} = Q_{sc} I,
\end{equation*}
where 
\begin{equation*}
I=1.83\times 0.216\frac{B^{A-1}}{(1-B)^A} \int_B^1
\left(\frac{1-y}{y}\right)^A \ln\left(\frac{33y}{B}\right) d y, 
\end{equation*}
with  
\begin{equation*}
A = \frac{W_s}{\kappa u_*},\quad u_*=\sqrt{\frac{\tau_{cw}}{\rho}},\quad B = k_s/h.
\end{equation*}

\item \texttt{ICF = 5}: Soulsby-van Rijn formula, the transport rate due to the combined action of
waves and current is provided by the following equation:
\begin{equation*}
Q_{b,s} = A_{b,s} U\left[ \left( U^2+2\frac{0.018}{C_D} U_0^2\right)^{0.5}-U_{cr}\right]^{2.4}. 
\end{equation*}
This formula can be applied to estimate both components of the total sand
transport rate (bedload $Q_b$ and suspension $Q_s$), and it is suitable for rippled bed
regimes. The bedload and suspended load coefficients, $A_{b,s}$ are computed:
\begin{align*}
A_b &= \frac{0.005 h \left(d_{50}/h\right)^{1.2}}{\left((s-1)gd_{50}\right)^{1.2}} \\
A_s &= \frac{0.012 d_{50}D_*^{-0.6}}{\left((s-1)gd_{50}\right)^{1.2}},
\end{align*}
where $U$ is the depth-averaged current velocity, $U_0$ is the RMS orbital velocity of waves, and 
$C_D$ is the quadratic drag coefficient due to current alone. This formula has been validated assuming a rippled bed roughness with $k_s = 0.18$ m. The critical entrainment velocity $U_{cr}$ is given by:
\begin{equation*}
U_{cr} = \left\{\begin{array}{ll}
\displaystyle
0.19 d_{50}^{0.1}\log_{10}\left(\frac{4h}{D_{90}}\right), & \quad \text{if } 0.1 mm\leq d_{50}\leq 0.5 mm \\
\displaystyle
8.5 d_{50}^{0.6} \log_{10}\left(\frac{4h}{D_{90}} \right), & \quad \text{if } 0.5 mm\leq d_{50}\leq 2.0 mm.
\end{array}
\right.
\end{equation*}
The diameter $D_{90}$, characteristic of the coarser grains, is specified with the keyword \texttt{D90}. The validity range for the Soulsby-van Rijn formula is $h = 1-20$ m, $U = 0.5-5$ms$^{-1}$, and $d_{50}=0.1-2$mm.

\item \texttt{ICF = 8}: the Bailard's formula is based on an energetic approach. The bedload and
the suspended load components of the sand transport rate are expressed
respectively as the third- and fourth-order momentum of the near-bed
time-varying velocity field $\overrightarrow{U(t)}$, as follows

\begin{align*}
Q_b &= \frac{f_{cw}}{g(s-1)} \frac{\epsilon_c}{\tan \varphi}
\left\langle \left| \overrightarrow{U} \right|^2 \overrightarrow{U}
\right\rangle \\
Q_s &= \frac{f_{cw}}{g(s-1)} \frac{\epsilon_s}{W_s} \left\langle
\left| \overrightarrow{U} \right|^3 \overrightarrow{U} \right\rangle 
\end{align*}
with $f_{cw}$ the friction coefficient which accounts
for wave-current interactions, $\epsilon_s  = 0.02$, $\epsilon_c = 0.1$ empirical
factors, $\varphi$ sediment friction angle ($\tan \varphi = 0.63$)  and $<\cdot>$ time-averaged over a wave-period.

Under combined wave and currents, the time-varying velocity vector $\overrightarrow{U(t)}$
is composed of a mean component $U_c$ and an oscillatory component of
amplitude $U_0$, assuming linear waves; $\phi$ is the angle between the
current and the wave direction, $\phi_c$ is the angle between the mean
current direction and the $x-$axis, and $\phi_w$ is the angle between the wave
direction of propagation and the x-axis. The time-varying velocity field verifies:
\begin{equation*}
\overrightarrow{U(t)} =U_c \exp^{i\phi_c} + U_0 \cos \omega t \exp^{i\phi_w}. 
\end{equation*}

For the third-order term, one can derive an analytical
expression in the general case, whereas for the fourth-order momentum, we
have to assume the waves and currents to be co-linear ($\phi = 0$), in order
to find an analytical expression. For the general case of a non-linear
non-colinear wave, we would have to integrate numerically the fourth-order
velocity term~\cite{Camenen}. This method is however not
efficient.

Third-order term:
\begin{align*}
\left\langle\left| \overrightarrow{U} \right|^2 \overrightarrow{U}
\right\rangle &= \overrightarrow{U_c}(U_c^2+\frac{1}{2} U_0^2)+
\overrightarrow{U_0} (\overrightarrow{U_c}\cdot\overrightarrow{U_0})\\
\left\langle \left| \overrightarrow{U} \right|^2 \overrightarrow{U}
\right\rangle_x &= \left[ U_c^3 + U_c U_0^2 (1+\frac{1}{2} \cos
2\phi )\right] \cos \phi_c -\frac{1}{2} U_c U_0^2\sin 2\phi \sin
\phi_c\\
\left\langle \left| \overrightarrow{U} \right|^2 \overrightarrow{U}
\right\rangle_y &= \left[ U_c^3 + U_c U_0^2 (1+\frac{1}{2} \cos
2\phi )\right] \sin \phi_c +\frac{1}{2} U_c U_0^2\sin 2\phi \cos
\phi_c
\end{align*}

Fourth-order term:
\begin{align*}
\left\langle \left| \overrightarrow{U} \right|^3 \overrightarrow{U}
\right\rangle_x = \frac{1}{8} (24 U_0^2 U_c^2 + 8 U_c^4 +3U_0^4)\cos \phi_c\\ 
\left\langle \left| \overrightarrow{U} \right|^3 \overrightarrow{U}
\right\rangle_y = \frac{1}{8} (24 U_0^2 U_c^2 + 8 U_c^4 +3U_0^4)\sin \phi_c
\end{align*}

\item \texttt{ICF = 9}: the Dibajnia and Watanabe (1992) formula is an unsteady formula, which
accounts for inertia effects. The sand transport rate is calculated by:
\begin{equation}\label{eq:Dibajnia}
\frac{\overrightarrow{Q_s} }{W_s d_{50}} =\alpha \frac{\overrightarrow{
\Gamma} }{\left| \Gamma \right|^{1-\beta}}, 
\end{equation}
with $\alpha=0.0001$ and $\beta=0.55$ empirical model
coefficients and $\overrightarrow{\Gamma}$ is the difference between the amounts of sediments transported
in the onshore and offshore directions. This formula is used to estimate the
intensity of the solid transport rate, the direction is assumed to follow
the mean current direction. 

In the wave direction, the time-varying velocity is given by:
\begin{equation*}
U(t)=U_c \cos \phi + U_0 \sin \omega t, 
\end{equation*}
$r$ is defined by the asymmetry parameter:
\begin{equation*}
r=\frac{U_0}{U_c\cos \phi}. 
\end{equation*}

The wave cycle is decomposed into two parts:
\begin{enumerate}
\item During the first part of the wave-cycle ($0 <  t < T_1$), 
the velocities are directed onshore ($U(t) > 0$).
\item During the second part ($T_2 = T-T_1$), the
velocities are negatives ($U(t) <0$).
\end{enumerate}

The sand transport rate in the wave direction is:
\begin{equation*}
\Gamma_{X'} = \frac{u_1T_1(\Omega_1^3+\Omega_2^{'3})-u_2T_2(\Omega_2^3+\Omega_1^{'3})}{(u_1+u_2)T}
\end{equation*}
where $\Omega_1$ and $\Omega_2$ represent the amount of sand
transported in the onshore and offshore directions which will
be deposited before flow reversal, respectively, $\Omega'_1$ and $\Omega'_2$
represent the remaining part which stay in suspension after flow reversal.
The inertia of sediment depends on the ratio $d_{50}/w_s$, and
represented by parameter $\omega_i$:
\begin{equation*}
\omega_i = \frac{u_i^2}{2(s-1)gW_s T_i}, 
\end{equation*}
with 
\begin{equation*}
u_i^2 = \frac{2}{T_i} \int_{u>\text{or}<0} u^2 d t, 
\end{equation*}
and two different cases:
\begin{enumerate}
\item All sediment is deposited before flow reversal $\omega_i\leq \omega_{crit}$
\begin{equation*}
\Omega_i=\omega_i\frac{2W_s T_i}{d_{50}}, \quad \Omega'_i = 0
\end{equation*}

\item Part of the sediment stays in suspension after flow reversal $\omega_i \geq \omega_{crit}$
\begin{equation*}
\Omega_i = \omega_{crit} \frac{2W_s T_i}{d_{50}}, \quad \Omega'_i=(\omega_i-\omega_{crit})\frac{2W_sT_i}{d_{50}}.
\end{equation*}

\end{enumerate}

The critical value of $\omega_{crit}$ is calculated as function of the
wave-current non-dimensional friction:
\begin{equation*}
\Theta_{cw} = \frac{<\tau_{cw} >}{\rho}\frac{1}{(s-1)gd_{50}}. 
\end{equation*}

According to Temperville et Guiza~\cite{2000}: 

\begin{equation*}
\omega_{crit} = \left\{\begin{array}{ll}
\displaystyle
0.03, & \quad\text{if }\Theta_{cw}\leq 0.2 \\
1-\sqrt{1-((\Theta_{cw}-0.2)/0.58)^2}, & \quad\text{if } 0.2\leq\Theta_{cw}\leq 0.4 \\  
\displaystyle
0.8104\sqrt{\Theta_{cw}} - 0.4225, & \quad\text{if } 0.4\leq\Theta_{cw}\leq 1.5\\  
\displaystyle
0.7236\sqrt{\Theta_{cw}} - 0.3162, & \quad\text{otherwise}
\end{array}
\right.
\end{equation*}

\end{itemize}

\begin{bclogo}[couleur = blue!10, arrondi = 0.10, logo = \bcattention]{\textsf{Attention}}
In \sisyphe, the solid discharge is assumed to be oriented in the direction of the mean current.
The possible deviation of the transport in the direction of waves, which can
be important in the near shore zone due to non-linear effects, is not
accounted for.
\end{bclogo}
