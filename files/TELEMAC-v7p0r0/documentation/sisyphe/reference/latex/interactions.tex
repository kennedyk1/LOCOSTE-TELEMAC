\section{Flow-sediment interactions}\label{sec:interactions}
\subsection{Sediment and fluid properties}
Fine sediment particles of grain size $d_{50} < 60\mu$m\footnote{$d_{50}$ is the grain size with 50\% of the material finer by weight} present complex
cohesive properties which affect the sediment transport processes. For non-cohesive sediments (median diameter $d_{50} >60 \mu$m), the grain diameter and grain density $\rho_s$ are the key parameters which determine its resistance to erosion and sediment transport rate.

For cohesive sediments ($d_{50} < 60 \mu$m), the grain diameter is
no longer the key sediment parameter: the settling velocity now depends on
the concentration of the sediment, whereas the critical bed shear strength depends on the
consolidation state of the sediment bed, see~\cite{}.

In this section, we consider uniform, non-cohesive sediments characterized by one single value for the grain size $d_{50}$ and grain density $\rho_s$, with $\rho_s=$2650 kg\,m$^{-3}$ which can be transported both as bed-load and suspended load.

\medskip
\begin{bclogo}[couleur=blue!10,arrondi=0.1, logo=\bcinfo]{Keywords}
The physical properties of the sediment are always defined in the \sisyphe steering file using the following keywords:
\begin{itemize}
\item {\ttfamily COHESIVE SEDIMENTS} ({\ttfamily = NO} , default option)
\item {\ttfamily NUMBER OF SIZE-CLASSES OF BED MATERIAL} ({\ttfamily NSICLA=
1}, default option) 
\item {\ttfamily SEDIMENT DIAMETERS} ($d_{50} > 60\mu$m for non-cohesive sediments)
\item {\ttfamily SEDIMENT DENSITY} ($\rho_s = 2650$ kgm$^{-3}$, default value)
\item {\ttfamily SETTLING VELOCITIES} (The default value is not given. If the user does not
give a value, the subroutine {\ttfamily vitchu.f} is used:
Stokes, Zanke or Van Rijn formulae depending on the grain size)
\end{itemize}
The physical properties of the fluid are defined by:
\begin{itemize}
\item {\ttfamily WATER VISCOSITY} ($\nu= 1.0\times 10^{-6}$ m$^2$s$^{-1}$, by default)
\item {\ttfamily WATER DENSITY} ($\rho=1000$ kgm$^{-3}$, by default)
\item {\ttfamily GRAVITY ACCELERATION} ($g=9.81$ ms$^{-2}$)
\end{itemize}
\end{bclogo}

\subsection{Bed shear stress}
\subsubsection{Hydrodynamic model}
The current-generated bed shear stress is used in both the shallow water momentum
equation as well as the bottom boundary condition for the velocity profile. 
When \sisyphe is coupled with \teldd, the bed shear stress term $\tau_{0}=(\tau_{x},\tau_{y})^T$
is calculated at each time step from the classical quadratic dependency on the depth--averaged velocity:
\begin{equation}\label{eq:Cd}
\tau_{x,y} =\frac{1}{2} \rho C_d (u,v)^T|\mathbf u|, 
\end{equation}
where $(u,v)^T$ are the depth-averaged
velocity components along the $x-$ and $y-$ Cartesian directions, respectively with transpose $(\cdot)^T$; $|\mathbf u| = \sqrt{{u}^2+{v}^2}$ the velocity module; and friction coefficient $C_d$.

When \sisyphe is coupled with \telddd, the bed shear stress is aligned
with the near bed velocity in order to account for possible vertical flow deviations. The magnitude of the bed shear stress is still related
to the depth-averaged velocity, except if the Nikuradse friction law is
applied. In that case, the friction velocity u$_{*}$, defined by
\begin{equation}\label{eq:us}
\tau _0 = \rho u_*^2 
\end{equation}
is related to the near bed flow velocity $u(z_1)$ by a logarithmic
velocity profile:
\begin{equation}\label{eq:velprof}
u(z_1) = \frac{u_*}{\kappa} \ln\left(\frac{z_1}{z_0}\right)  
\end{equation}
\nomenclature{$z_0$}{vertical distance from a rough boundary}  
where $z_0$ is the vertical distance from a rough boundary, expressed as a function of the Nikuradse bed roughness ($z_0 = k_s/30$), with $k_s$ the grain roughness height, $z_1$ is the near bed distance measured along the vertical coordinate $z$, and aligned against the direction of the acceleration of gravity of magnitude $g$. In \telddd, $z_1$ is taken equal to the velocity in the bottom computational layer; and $\kappa = 0.4$ is the von K\'arm\'an constant. For flat beds, the roughness height has been shown to be approximately $k_s\approx 3 d_{50}$~\cite{JULIEN}.

The direction of the bed shear stress and resulting bedload transport rate is
assumed to be in the direction of the depth-averaged velocity in \sisyphe
(alone or internally coupled with \teldd). When \sisyphe is internally
coupled to \telddd, the bed shear stress and resulting transport rate are
assumed to be in the direction of the near bed velocity. The 3D model gives
a more accurate estimate of the bottom friction, since it accounts for a
possible vertical deviation of the current.

\begin{bclogo}[couleur = blue!10, arrondi = 0.10, logo = \bcattention]{\textsf{\sisyphe coupled with \teldd}}
When the model is coupled with \teldd, the values of the friction coeffients (and therefore, 
the bed shear stress) are provided by \teldd. The depth-averaged bed shear stress and resulting bedload transport rates are assumed to be in the direction of the mean flow velocity, except when the sediment transport
formulation accounts for:
\begin{itemize}
\item deviation correction (bed slope effect)
\item secondary currents 
\end{itemize} 
\end{bclogo}

Both bed shear stress and bed load transport rate are aligned with the near
bed velocity. This 3D approach is more physical, and takes into account
possible recirculation and veering of the flow, such that corrections for
secondary currents for example are no longer necessary.

\subsubsection{Uncoupled model}
The quadratic friction coefficient $C_d$ which is used to calculate the
total bed shear stress can be calculated based on the selected
friction law. Different options, which are consistent with the \teldd
options, are available in \sisyphe and depend on the choice of the keywords \texttt{LAW FOR
BOTTOM FRICTION} and on the value of the \texttt{FRICTION COEFFICIENT}:

\begin{itemize}
\item  Ch\'{e}zy coefficient $C_h$ (\texttt{KFROT = 2})
\begin{equation}\label{eq:chezy}
C_d = \frac{2g}{C_h^2} 
\end{equation}
\item  Strickler coefficient $S_t$ (\texttt{KFROT = 3})
\begin{equation}\label{eq:strickler}
C_d = \frac{2g}{S_t^2}\frac{1}{h^{1/3}} 
\end{equation}
\item  Manning friction $M_a$ (\texttt{KFROT = 4})
\begin{equation}\label{eq:manning}
C_d =\frac{2g}{h^{1/3} } M_a^2 
\end{equation}
\item  Nikuradse bed roughness $k_s$ (\texttt{KFROT = 5})
\begin{equation}\label{eq:nikuradse}
C_d  = 2\left[\frac{\kappa}{\log(\frac{12h}{k_s})}\right]^2,  
\end{equation}

\end{itemize}

\medskip
\begin{bclogo}[couleur=blue!10,arrondi=0.1, logo=\bcinfo]{Keywords}
In the \sisyphe steering file, the total bed shear stress is calculated based
on:
\begin{itemize}
\item {\ttfamily LAW OF BOTTOM FRICTION} ({\ttfamily KFROT =3}, default option)
\item {\ttfamily BOTTOM FRICTION COEFFICIENT} ({\ttfamily St=50}, default value)
\end{itemize}
\end{bclogo}

Similar keywords are available in both \teldd and \telddd in the case of
internal coupling.


\subsubsection{Role of bed forms}
A natural sediment bed is generally covered with bed forms (length $\lambda_d$
and height $\eta_d$). The presence of bed forms greatly modifies the boundary
layer flow structure, with the formation of recirculation cells and
depressions in the lee of bed forms.
Depending on the flow and sediment transport rates, the size of bed
forms ranges from a few centimeters for ripples to a few tens of meter for
mega-ripples. The dimension of dunes scales with the water depth $h$, such that $\eta_d
\approx 0.4 h$ and  $\lambda_d\approx 6-10 h$. 
In most of cases, large scale models do not resolve the small to medium scale
bed forms (ripples, mega-ripples) which need therefore to be parameterized
by an increased friction coefficient. The total bed shear stress is expressed 
as the sum of two components: 
\begin{equation}\label{eq:taupp}
\tau_0 = \tau_0' + \tau_0''
\end{equation}
where $\tau_0$ is the total bed shear stress, $\tau_0'$ is the grain (or skin) shear stress, 
and $\tau_0''$ is the form shear stress.
The local skin friction component determines the bedload transport rate and
the equilibrium concentration for the suspension. The total friction
velocity determines the (spatially averaged along bedforms) turbulence eddy
viscosity/diffusivity vertical distribution in 3D models, and therefore
determines both the velocity vertical profile and the mean concentration profile.

\subsection{Skin friction correction}
\subsubsection{Total bed shear stress decomposition}
Bedload transport rates are calculated as a function of the local skin friction
component $\tau'$. The total bed shear stress issued from the hydrodynamics
model needs to be corrected in the morphodynamics model as follows:

\begin{equation}\label{eq:skinfriction}
\tau'= \mu \tau_0, 
\end{equation}
where $\mu$ is a correction factor for skin friction. Physically, the skin bed roughness should be smaller than the total bed
roughness (i.e. $\mu \leq 1$). However, in most cases the hydrodynamic
friction does not represent the physical bottom friction: the coefficient is
generally used as a calibration coefficient in hydrodynamics models. It is
adjusted by comparing simulation results with observations of the time-varying free
surface and velocity field. Therefore, its model value integrates various
neglected processes (side wall friction, possible errors in the bathymetry
and input data). Under those conditions, a correction factor $\mu > 1$ can be admitted.

Different methods are programmed in \sisyphe in order to calculate the bedform
correction factor $\mu$, according to the keyword \texttt{SKIN FRICTION CORRECTION}:

\begin{itemize}
\item \texttt{ICR = 0}: no correction, the total friction issued from
\tel is directly used for sand transport calculations ($\mu=1$).
\item \texttt{ICR = 1}: the skin roughness is assumed to be proportional
to the sand grain diameter like in the case of flat beds ($k_s' \sim d_{50}$). The proportionality coefficient is specified by the keyword \texttt{RATIO BETWEEN SKIN FRICTION AND MEAN DIAMETER}, defined as:
\begin{equation}\label{eq:icr1}
\mu =\frac{C_d'}{C_d}, 
\end{equation}
where $C_d$ et $C_d'$ are both quadratic friction coefficients related to total friction and skin friction,respectively. $C_d$ is obtained from \teldd or \telddd
and $C_d'$ is calculated from $k_s'$, as follows:
\begin{equation}\label{eq:ksp}
C_d' = 2\left[\frac{\kappa}{\log(\frac{12h}{k_s'})}\right]^2 
\end{equation}
\item \texttt{ICR = 2} the bedform predictor is used to calculate the
bedform roughness $k_r$ in order to account for the effect of ripples. Both $k_r$
and $k_s'$ should influence the transport rates. It is assumed that:
\begin{equation}\label{eq:mu}
\mu =\frac{C_d'^{0.75} C_r^{0.25}}{C_d}, 
\end{equation}
where the quadratic friction $C_r$ due to bedforms is calculated as a
function of $k_r$. For currents only, the ripple bed roughness is function of the mobility number, see~\cite{vanRijn07}:

\begin{equation*}
k_r =\left\{\begin{array}{ll}
d_{50}(85-65\tanh(0.015(\Psi-150))) & \text{for\,} \Psi<250\\
20 d_{50} & \text{otherwise} 
\end{array}
\right.
\end{equation*}
with $\Psi =U^2/(s-1)gd_{50}$.
\end{itemize}

\medskip
\begin{bclogo}[couleur=blue!10,arrondi=0.1, logo=\bcinfo]{Keywords}
The option selected for the skin friction correction is based on keywords:
\begin{itemize}
\item {\ttfamily SKIN FRICTION CORRECTION} ({\ttfamily ICR=0}, default option)
\item {\ttfamily RATIO BETWEEN SKIN FRICTION AND MEAN DIAMETER} ({\ttfamily KSPRATIO=3}, default value)
\end{itemize}
\end{bclogo}

\subsection{Bed roughness predictor}
\subsubsection{Total bed roughness (uncoupled)}
Different options are programmed in \sisyphe to predict the total bed
roughness through the associated keywords {\ttfamily BED ROUGHNESS PREDICTION} and {\ttfamily BED ROUGHNESS PREDICTOR OPTION}. It is recalled that the bed friction option of \sisyphe is not
used in the case of internal coupling with \teldd or \telddd.
\begin{itemize}
\item For {\ttfamily IKS = 1}: the bed is assumed to be flat $k_s = k_s'= \alpha d_{50}$, with $\alpha$ a constant (assumed to be equal to $3$.
\item {\ttfamily IKS = 2}: for waves and combined waves and currents, bedform dimensions are calculated
as a function of wave parameters following the method of Wiberg and Harris~\cite{WibergHarris}. 
The wave-induced bedform bed roughness $k_r$ is calculated as a function of the wave-induced bedform 
height $\eta_r$: 
\begin{equation}
k_r = \max(k_s', \eta_r).
\end{equation}
\item {\ttfamily IKS = 3}: for currents only, the van Rijn's total bed roughness predictor~\cite{vanRijn07, Huybrechts10} has been implemented. 
The total bed roughness can be decomposed into a grain
roughness $k_s'$, a small-scale ripple roughness $k_r$, a mega-ripple 
component $k_{mr}$, and a dune roughness $k_d$:
\begin{equation}\label{eq:totalbedroughness}
k_s = k_s' + \sqrt{k_r^2 + k_{mr}^2 + k_d^2}. 
\end{equation}
Both small scale ripples and grain roughness have an influence on the
sediment transport laws, while the mega-ripples and dune roughness only
contribute to the hydrodynamic model (total friction). 

\end{itemize}





\subsection{Sediment transport}
\subsubsection{Bedload and suspended load}
When the current-induced bed shear stress increases above a critical
threshold value, sediment particles start to move as bedload, while the
finer particles are transported in suspension. Bedload occurs in a very thin high concentrated near-bed layer, where
inter-particle interactions develop. The suspended load is defined as the depth-integrated flux of sediment
concentration, from the top of the bedload layer up to the free surface.
For dilute suspension concentration values, clear flow concepts (turbulence diffusion, eddy
viscosity, logarithmic velocity profile) are considered to be valid.
The total sediment load $Q_t$ includes both a bedload $Q_b$ and suspended load $Q_s$:
\begin{equation}\label{eq:Qt}
Q_t  = Q_b + Q_s. 
\end{equation}

\subsubsection{Shields parameter}
The critical Shields number or dimensionless critical shear stress $\Theta_c$
is defined by:
\begin{equation}\label{eq:shields}
\Theta_c = \frac{\tau_c}{g(\rho_s -\rho)d_{50}} 
\end{equation}
where $\tau_c$ is the critical shear stress for sediment incipient motion. Values of $\Theta_c$
can be either specified in the \sisyphe steering file by use of keyword \texttt{SHIELDS PARAMETERS}
 or calculated by the model as a function of
non-dimensional grain diameter $D_* =d[(\rho_s/\rho-1)g/\nu^2]^{1/3}$. It is implemented in subroutine 
\texttt{init\_sediment.f}, see also~\cite{XXX}

\begin{equation*}
\frac{\tau_c}{g(\rho_s -\rho)d_{50}}=\left\{\begin{array}{ll}
0.24 D_*^{-1}, & D_* \leq 4 \\
0.14 D_*^{-0.64}, & 4 < D_* \leq 10 \\
 0.04 D_*^{-0.10}, & 10 < D_* \leq 20\\
0.013 D_*^{0.29}, & 20 < D_* \leq 150 \\
0.045, & 150 \leq D_* 
\end{array}
\right.
\end{equation*}

%\begin{align*}
%D_* \leq 4,\quad & \theta_c = 0.24 D_*^{-1} \\
%4 < D_* \leq 10,\quad & \theta_c = 0.14 D_*^{-0.64} \\
%10 < D_* \leq 20,\quad & \theta_c = 0.04 D_*^{-0.10} \\
%20 < D_* \leq 150,\quad & \theta_c = 0.013 D_*^{0.29} \\
%150 \leq D_*,\quad & \theta_c = 0.045,
%\end{align*}

where $\tau_c$ and $d$ are in N\,m$^{-2}$ and m, respectively. The default option (no Shields number given in steering file) is to calculate the shields parameter as a function of 
sand grain diameter (see logical \texttt{CALAC}).





